\documentclass[a4paper,12pt]{article}

%%% Работа с русским языком
\usepackage{cmap}					% поиск в PDF
\usepackage{mathtext} 				% русские буквы в формулах
\usepackage[T2A]{fontenc}			% кодировка
\usepackage[utf8]{inputenc}			% кодировка исходного текста
\usepackage[english,russian]{babel}	% локализация и переносы

\usepackage{amsmath,amsfonts,amssymb,amsthm,mathtools} % AMS
\usepackage{icomma} % "Умная" запятая: $0,2$ --- число, $0, 2$ --- перечисление

%%% Работа с картинками
\usepackage{graphicx} % Для вставки рисунков
\usepackage{wrapfig} % Обтекание рисунков текстом

%%% Работа с таблицами
\usepackage{array,tabularx,tabulary,booktabs} % Дополнительная работа с таблицами
\usepackage{longtable} % Длинные таблицы
\usepackage{multirow} % Слияние строк в таблице
\usepackage[overload]{empheq} % для рамки в формулах

%%% Страница
\usepackage{extsizes} % Возможность сделать 14-й шрифт
\usepackage{geometry} % Простой способ задавать поля
	\geometry{top=25mm}
	\geometry{bottom=35mm}
	\geometry{left=35mm}
	\geometry{right=20mm}


\begin{document}

\section{Теорема Лагранжа о среднем значении: следствия}

Первой областью, к которой мы применим TCЗ, будут графики, но, как мы увидим позже, эта теорема играет важную роль в анализе.

\begin{itemize}
	\item Если $f' > 0$, то $f$ -- возрастающая.
	\item Если $f' < 0$, то $f$ -- убывающая.
	\item Если $f' = 0$, то $f$ -- постоянная.
\end{itemize}

Мы сказали, что первые два утверждения верны, но не доказали. Теперь мы можем доказать их с помощью ТСЗ.

\textbf{Доказательство}: Теорема о среднем значении говорит нам, что:

\[
	\frac{f(b) - f(a)}{b-a} = f'(c) \notag
\]

Для некоторого $c$, лежащего между $a$ и $b$. В данном доказательстве мы примем такие $b$ и $a$, что $b > a$.
Мы напишем уравнение ТСЗ ''в обратном направлении'', так как мы хотим использовать производную $f'$ для получения информации о $f$.

Мы проведем некоторые действия над уравнением:
\begin{eqnarray}
f(a) - f(b) &=& f'(c)(b-a) \notag \\
f(b) &=& f(a)+f'(c)(b-a) \notag
\end{eqnarray}

Эта новая форма ТСЗ дает нам возможность проверить три упомянутых утверждения.

Поскольку $a<b$, $b-a>0$, то знак выражения $f'(c)(b-a)$ полностью зависит от знака выражения $f'(c)$.
\begin{itemize}
	\item Если $f'(с) > 0$, то $f(b) > f(a)$.
	\item Если $f'(с) < 0$, то $f(b) < f(a)$.
	\item Если $f'(с) = 0$, то $f(b) = f(a)$.
\end{itemize}

Эти утверждения могут показаться очевидными, однако это не так. Определение производной описано в терминах бесконечно малых. Это не совсем верное определение, так как эти бесконечно малые не имеют ничего общего с поведением функции в больших масштабах. Раньше мы утверждали, что отношение разностей приблизительно равно производной. Сейчас мы говорим, что оно точно равно производной (Несмотря на это мы не знаем, в какой момент следует взять производную.)
\end{document}	