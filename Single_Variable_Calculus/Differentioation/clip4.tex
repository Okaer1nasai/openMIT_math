\documentclass[a4paper,12pt, 3in, 4in]{article}

    \usepackage{pgf,tikz}
\usepackage{mathrsfs}
\usetikzlibrary{arrows}
%% Работа с русским языком
\usepackage{cmap}					% поиск в PDF
\usepackage{mathtext} 				% русские буквы в формулах
\usepackage[T2A]{fontenc}			% кодировка
\usepackage[utf8]{inputenc}			% кодировка исходного текста
\usepackage[english,russian]{babel}	% локализация и переносы

%% Дополнительная работа с математикой
\usepackage{amsmath,amsfonts,amssymb,amsthm,mathtools} % AMS

%% Заголовок
\begin{document}
    \section*{Наклон касательной к $a^x$}
    Мы определили функцию $M(a)$ как
    \[
        M(a) = \lim\limits_{\Delta x \to 0} \cfrac{a^{\Delta x} - 1}{\Delta x}
    \]
    Исходя из этого, мы можем сказать, что $\cfrac{d}{dx}a^x = M(a)a^x$. Чтобы понять, что из себя представляет производная $a^x$, нужно понимать, что из себя представляет $M(a)$; далее будем рассматривать $M(a)$ двумя разными способами. \par
    Во-первых, подставим $x = 0$ в определение производной
    \begin{align*}
        \cfrac{d}{dx} a^x \;\biggr{|}_{x=0} &= \lim\limits_{\Delta x \to 0} \cfrac{a^{x + \Delta x} -a^x}{\Delta x}\;\biggr{|}_{x=0} =  \lim\limits_{\Delta x \to 0} \cfrac{a^{0+\Delta x} - a^0}{\Delta x}= \\ &= \lim\limits_{\Delta x \to 0} \cfrac{a^{\Delta x} - 1}{\Delta x} = M(a)
    \end{align*}
    (другими словами, $\cfrac{d}{dx} a^x\biggr{|}_{x = 0} = M(a)a^0 = M(a)$). Итак, $M(a)$ --- это значение производной в точке $0$. \par
    Напомним, что значение производной в точке говорит нам о наклоне касательной к этому графику в этой точке. То есть $M(a)$ --- это угловой коэффициент касательной в точке 0 для графика функции $y = a^x$. \par
    Заметим, что форма графика функции $y = a^x$ зависит от выбора $a$, значит для разных $a$ мы будем получать разные касательные, а значит и разные $M(a)$. \par
    Поскольку $\cfrac{d}{dx}a^x = M(a)a^x$, единственное, что нам надо знать --- это значение углового коэффициента в точке 0, чтобы определить его для любой другой точки графика. \par
    Напомним, что для вычисления производной синуса нам пришлось постараться для вычисления предела $\lim\limits_{x \to 0} \cfrac{\sin x}{x}$. Это значение и есть производная синуса в точке 0, как показывает прямая проверка по определению. Для того, чтобы выяснить значение производнойв любой точке, нам нужно было вычислить её значение в нуле.\par
    Формула для $a^{x+\Delta x}$ проще, чем таковая для $\sin(x + \Delta x)$, так что первая часть вычислений производной была проще. Однако вычислить предел
    \[
    M(a) = \lim\limits_{\Delta x \to 0}\cfrac{a^{\Delta x} - 1}{\Delta x}
    \]
    у нас не получается. \par
    \definecolor{qqwuqq}{rgb}{0.,0.39215686274509803,0.}
\begin{tikzpicture}[line cap=round,line join=round,>=triangle 45,x=1.0cm,y=1.0cm]
\draw[->,color=black] (-3.264208744039613,0.) -- (7.220584827434731,0.);
\foreach \x in {-3.,-2.,-1.,1.,2.,3.,4.,5.,6.,7.}
\draw[shift={(\x,0)},color=black] (0pt,2pt) -- (0pt,-2pt) node[below] {\footnotesize $\x$};
\draw[->,color=black] (0.,-2.561931120324355) -- (0.,5.711683918618382);
\foreach \y in {-2.,-1.,1.,2.,3.,4.,5.}
\draw[shift={(0,\y)},color=black] (2pt,0pt) -- (-2pt,0pt) node[left] {\footnotesize $\y$};
\draw[color=black] (0pt,-10pt) node[right] {\footnotesize $0$};
\clip(-3.264208744039613,-2.561931120324355) rectangle (7.220584827434731,5.711683918618382);
\draw[line width=1.2pt,color=qqwuqq,smooth,samples=100,domain=-3.264208744039613:7.220584827434731] plot(\x,{1.7^((\x))});
\draw [domain=-3.264208744039613:7.220584827434731] plot(\x,{(--1.--0.5306282510621704*\x)/1.});
\draw (4.833683497615851,4.540198603369853) node[anchor=north west] {$M(a)$};
\draw (-2.839545317262021,-0.3507525877927562) node[anchor=north west] {(угловой коэффициент касательной к $a^x$ в точке 0)};
\draw (2.256415804069085,4.364475806082574) node[anchor=north west] {$a^x$};
\begin{scriptsize}
\draw[color=qqwuqq] (-3.1470602125147598,0.08855440542544224) node {$f$};
\draw[color=black] (-3.1470602125147598,-0.4093268535551826) node {$a$};
\end{scriptsize}
\end{tikzpicture}
\parДля нахождения предела $\lim\limits_{x \to 0} \cfrac{\sin x}{x}$ мы могли использовать свойства радианной меры угла и единичную окружность, но тут хороших методов нахождения точного значения углового коэффициента касательной для $a^x$ в точке $x=0$ у нас нет
\end{document}