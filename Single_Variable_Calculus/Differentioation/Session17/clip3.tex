\documentclass[a4paper,12pt, 3in, 4in]{article}

%% Работа с русским языком
\usepackage{cmap}					% поиск в PDF
\usepackage{mathtext} 				% русские буквы в формулах
\usepackage[T2A]{fontenc}			% кодировка
\usepackage[utf8]{inputenc}			% кодировка исходного текста
\usepackage[english,russian]{babel}	% локализация и переносы

%% Дополнительная работа с математикой
\usepackage{amsmath,amsfonts,amssymb,amsthm,mathtools} % AMS

%% Заголовок
\begin{document}
    \section*{$a^x$ и определение производной}
    Наша задача вычислить производную $\cfrac{d}{dx}a^x$. \par
    Для начала запишем производную по определению:
    \[
        \cfrac{d}{dx} a^x = \lim\limits_{\Delta x \to 0}\cfrac{a^{x + \Delta x} - a^{x}}{\Delta x}
    \]
    Пользуясь правилом $a^{x_1+x_2} = a^{x_1}a^{x_2}$, чтобы вынести $a^x$:
    \[ 
        \lim\limits_{\Delta x \to 0}\cfrac{a^{x + \Delta x} - a^{x}}{\Delta x} = \lim\limits_{\Delta x \to 0}\cfrac{a^xa^{\Delta x} - a^x}{\Delta x} = \lim\limits_{\Delta x \to 0}a^x\cfrac{a^{\Delta x} - 1}{\Delta x}
    \]
    Поскольку мы, считая предел, полагаем, что $a$ и $x$ фиксированы, в то время как $\Delta x$ у нас изменяется, мы можем вынести $a^x$ за знак предела: 
    \[
        \cfrac{d}{dx} a^x = a^x\lim\limits_{\Delta x\to 0}\cfrac{a^{\Delta x} - 1}{\Delta x}
    \]
    Мы неплохо начали. Посмотрим, что у нас есть. Мы видим, что наша производная $\cfrac{d}{dx} a^x$--- это $a^x$, умноженная на некоторый множитель, значение которого мы пока не знаем. Обозначим его как $M(a)$. 
    \[
        M(a) = \lim\limits_{\Delta x \to 0}\cfrac{a^{\Delta x} - 1}{\Delta x}
    \]
    То есть, пользуясь определением, можем переписать производную как
    \[
        \cfrac{d}{dx}a^x  = M(a)a^x
    \]
\end{document}