\documentclass[a4paper,12pt, 3in, 4in]{article}

%% Работа с русским языком
\usepackage{cmap}					% поиск в PDF
\usepackage{mathtext} 				% русские буквы в формулах
\usepackage[T2A]{fontenc}			% кодировка
\usepackage[utf8]{inputenc}			% кодировка исходного текста
\usepackage[english,russian]{babel}	% локализация и переносы

%% Дополнительная работа с математикой
\usepackage{amsmath,amsfonts,amssymb,amsthm,mathtools} % AMS

%% Заголовок
\begin{document}


\section*{Работа с показателем степени}
    Начнём с определения <<основания>> числа $a$. Это число должно быть положительным, и будем считать, что $a>1$ для упрощения построения графиков. \par
    Вспомним для начала, что мы знаем о показатеях степени? Для натуоальных чисел степень числа можно определить так:
    \[ 
        a^1 = a;\; a^2 = a\cdot a;\; \ldots
    \]
    В общем случае,
    \[
        a^{x_1+x_2} = a^{x_1}a^{x_2}
    \]
    Из этих свойств мы можем вывести
    \[ 
        (a^{x_1})^{x_2} = a^{x_1x_2}
    \]
    и легко определить показательную функцию для любого натурального $n$. Для целых отрицательных мы видим из вышестоящих свойств, что $a^{m}a^{-m} = a^{m-m}= 1\Rightarrow a^{-m} = \cfrac{1}{a^{m}}$.\par
    Однако мы хотим определить $a^x$ для любого действительного $x$, не только для целых. Начнём с определения его для рациональных $x$:
    \[
        a^{\cfrac{p}{q}} = \sqrt[q]{a^p}\;\text{(где $p$ и $q$ натуральные)}
    \]
    Поскольку $a^1 = a^{1/2 + 1/2} = a^{1/2}a^{1/2} = \sqrt{a}\cdot \sqrt{a} = a$, это определение кажется вполне обоснованным.\par
    Нам осталось только расширить это определение для иррациональных $x$. Для этого будем <<заполнять>> промежутки между рациональными числами, чтобы функция получилась непрерывной. Это именно то, что делает ваш калькулятор, когда вы просите его посчитать значение $2^\sqrt{2}$ или $3^{\pi}$, вы не получите точного ответа, вместо этого устройство выдаст вам рациональное число (в виде десятичной дроби), близкое к точному значению. \par
    Немного погодя, мы нарисуем график этой функции, и узнаем, как показательная функция устроена
\end{document}
